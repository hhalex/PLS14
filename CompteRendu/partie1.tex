\section{Introduction}

\par Le technologies d'aujourd'hui permettent de stocker toujours plus de données, pour des coûts raisonnables. Il y a encore quelques années, les volumes échangés étaient limités par la vitesse de transmission et la capacité de stockage des médias disponibles. Ces contraintes ayant été outrepassées, l'accumulation de données est désormais telle qu'il devient difficile de les utiliser telles quelles. \emph{Big Data} est le terme désignant cette masse conséquente, et englobe même tous les défis qui y sont liés, parmi lesquels l'analyse, la capture, le stockage...
\par Un des défis est justement d'appliquer des algorithmes sur ces données, mais de manière intelligente. Il ne s'agit pas en effet de considérer un seul ordinateur pour effectuer des calculs aussi conséquents, mais plutôt de répartir le traitement en plusieurs sous taches assignées à plusieurs unités de calcul indépendantes. De cette manière, au lieu de se limiter à la puissance d'une seule unité, on considère l'ensemble des ressources disponibles sur le réseau.
\par Des couches destinées à la gestion des ressources disponibles existent, et \emph{OAR} en fait partie. Elles permettent notamment de vérifier leur disponibilité, et d'y lancer des traitements. Notre but sera ainsi de fournir une surcouche permettant d'interagir avec les ressources disponibles sur le réseau à l'aide d'\emph{OAR} de manière transparente, et d'y lancer des taches pour traiter de gros fichiers de données. Cette surcouche sera en fait une interface graphique codée en JAVA, et sera suffisamment modulaire pour permettre de changer aisément l'algorithme de traitement des données. Pour le calcul distribué, le framework \emph{Hadoop} sera utilisé.

%%% Local Variables: 
%%% mode: latex
%%% TeX-master: "CompteRendu"
%%% End: 
