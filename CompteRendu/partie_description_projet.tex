\section{Description du projet}

\par Le monde de l'IT fait actuellement parlé de lui à travers le Big Data. Google Trends rend compte de l'explosion de l'emploi de ce terme depuis fin 2010. Mais concrètement, quelle est l'utilité de ces technologies \emph{Big Data} dont tout le monde parle ? Les données produites par chacun au jour le jour représentent le nouvel Eldorado des sociétés qui voient en ces données l'occasion d'optimiser les process, d'améliorer l'expérience utilisateur, de proposer de nouveaux services ou encore, à titre d'exemple, d'aider la police à traquer les délinquants (\emph{L.A.}, \emph{USA}). Les applications \emph{Big Data} sont nombreuses et s'appuient pour partie sur les données open source mises en ligne par les gouvernements, les entreprises et les particuliers.

\par La deuxième année d'études à Supélec est ponctuée par la réalisation d'un projet de développement logiciel et d'un projet de conception chacun d'une durée d'une séquence (une année académique de Supélec est composée de quatre séquences).

\par Le choix des sujets est libre. C'est pourquoi nous avons choisi de nous intéresser aux technologies \emph{Big Data} et plus particulièrement à l'environnement Hadoop. Cet outil, très en vogue actuellement dans le monde de l'IT apparaît comme la référence dans le domaine.

\par Compte tenu de nos ambitions personnelles, notre projet prendra la forme d'un projet long, s'étalant donc sur deux séquences, et chapeauté par Mr Stéphane Vialle, professeur en Informatique à Supélec.

\par Notre projet s'articule en deux grandes parties. Premièrement, il s'agit d'une part de coder une interface graphique utilisateur en \emph{Java} à travers laquelle un utilisateur est en mesure de se connecter aux serveurs de Supélec ; d'autre part, il s'agit de permettre à l'utilisateur d'allouer une certaine puissance de calcul au sein des clusters Supélec (Skynet, Cameron et InterCel). Puissance qui, dans une deuxième partie, sera utilisée afin d'exécuter des applications Hadoop codées au préalable. Cette dernière partie se concentrera exclusivement sur la prise en main d'Hadoop et sera notamment l'occasion de travailler avec Mr Frédéric Pennerath, Professeur en Informatique à Supélec et chercheur en \emph{Data Mining}.


%%% Local Variables: 
%%% mode: latex
%%% TeX-master: "CompteRendu"
%%% End: 
