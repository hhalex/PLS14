
\section{Préliminaires}
\label{sec:preliminaires}

\par Avant de commencer concrètement le projet, une certaine organisation s'impose, ainsi qu'une énumération de certaines connaissances nécessaires pour la suite. 

\subsection{Gestion de version avec Git}
\label{sec:gestion-de-version}

\subsubsection{Utilité}
\label{sec:utilite}

\par Ce projet logiciel va nécessiter la coopération de plusieurs personnes sur des fichiers communs, l'utilisation d'un gestionnaire de version s'impose. 
\par Il s'agit de pouvoir gérer les interventions simultanées de plusieurs opérateurs sur un même fichier, et d'enregistrer progressivement les modifications apportées, et pouvoir revenir dessus si nécessaire. Ce type de système est massivement utilisé aujourd'hui, pour les projets conséquents, ou impliquant la coopération de plusieurs personnes.
\par Git est un des gestionnaires de version les plus récents, créé par Linus Torvalds afin de pouvoir gérer l'évolution du kernel Linux. Il est décentralisé, cequi veut dire que chaque intervenant possède une version du projet localement, et doit se synchroniser avec le projet de référence avant de pouvoir commencer à travailler (afin de bénéficier des modifications les plus récentes). Github.com met à disposition une plateforme permettant d'héberger facilement des projets versionnés avec Git, et c'est elle que nous utiliseront.

\par Comme \emph{OAR}, ce logiciel s'utilise en ligne de commande.

\subsubsection{Création d'un projet}
\label{sec:creation-dun-projet}

\par Dans notre cas, le projet a été créé avec les fonctionnalités intégrées de Github, mais en général, pour initialiser un projet git, on se place dans le répertoire du projet, à sa racine, puis on tape dans le terminal :
\begin{verbatim}
> git init
\end{verbatim}

\par Dans le cas d'un projet existant, on note l'adresse à laquelle le projet est disponible, par exemple \texttt{https://github.com/hhalex/PLS14}, et on tape dans le terminal:
\begin{verbatim}
> git clone https://github.com/hhalex/PLS14
\end{verbatim}

\par On verra normalement apparaître un dossier \texttt{.git} auquel il ne faudra surtout pas toucher puisqu'il s'agit des versions enregistrées par Git.

\subsubsection{Enregistrement des modifications}
\label{sec:enreg-des-modif}

\par En travaillant sur le projet, on peut enregistrer régulièrement les modifications localement. On effectue ainsi des points de sauvegarde, qu'on peut nommer explicitement afin de pouvoir s'y retrouver plus tard.
\par Après avoir apporté quelques modifications, placez-vous dans le répertoire du projet, à la racine, puis tapez :
\begin{verbatim}
> git status
\end{verbatim}
 \par Cette commande va afficher tout d'abord les fichiers nouveaux/modifiés qui sont sélectionnés pour l'enregistrement, puis les fichiers nouveaux/modifiés non sélectionnés. 

\begin{verbatim}
Alex > 
# On branch master
# Changes not staged for commit:
#   (use "git add <file>..." to update what will be committed)
#   (use "git checkout -- <file>..." to discard changes in working directory)
#
#	modified:   CompteRendu/page_garde.tex
#	modified:   CompteRendu/partie1.tex
#	modified:   CompteRendu/partie2.tex
#
# Untracked files:
#   (use "git add <file>..." to include in what will be committed)
#
#	CompteRendu/.#partie2.tex
#	CompteRendu/CompteRendu.aux
#	CompteRendu/CompteRendu.idx
#	CompteRendu/CompteRendu.log
#	CompteRendu/CompteRendu.pdf
#	CompteRendu/CompteRendu.synctex.gz
#	CompteRendu/CompteRendu.toc
#	CompteRendu/auto/
#	CompteRendu/page_garde.aux
#	CompteRendu/partie1.aux
#	CompteRendu/partie2.aux
#	CompteRendu/partie3.aux
#	CompteRendu/partie4.aux
#	CompteRendu/partie5.aux
#	CompteRendu/styles.aux
no changes added to commit (use "git add" and/or "git commit -a")
\end{verbatim}

\par Dans l'exemple précédent, aucun fichier n'est sélectionné pour le prochain enregistrement, appelé \emph{commit} dans le jargon de Git. On voit néanmoins apparaître 3 fichiers modifiés : \texttt{page\_garde.tex}, \texttt{partie1.tex}, \texttt{partie2.tex} encore non ajoutés à la liste des fichiers pour le prochain commit. Toute une série de fichiers divers apparaissent comme \emph{untracked}, dans notre cas il ne s'agit que de fichiers temporaires utiles pour la compilation du présent document. 

\par Pour sélectionner les fichiers à ajouter au prochain enregistrement, tapez :
\begin{verbatim}
> git add fichier 
\end{verbatim}

\par On peut aller plus vite en donnant le nom du \texttt{dossier} pour ajouter tous les fichiers contenus dans \texttt{dossier}. Par exemple, si on avait voulu enregistrer tous les fichiers temporaires générés par la compilation du compte rendu, on aurait pu taper :
\begin{verbatim}
> git add CompteRendu
\end{verbatim}

\par Maintenant que les fichiers intéressants ont été sélectionnés, on peut procéder à l'étape d'enregistrement. Toujours au même endroit dans le dossier du projet, tapez :
\begin{verbatim}
> git commit -m "description du commit"
\end{verbatim}

\par Par exemple :
\begin{verbatim}
 > git commit -m "[CompteRendu] Modification de la page de garde"
[master afe15b1] [CompteRendu] Modification de la page de garde
 1 file changed, 10 insertions(+), 5 deletions(-)
\end{verbatim}

\par Vous pouvez maintenant vérifier que votre commit est bien là en tapant :
\begin{verbatim}
> git log
\end{verbatim}

\par Dans notre exemple :

\begin{verbatim}
> git log
commit afe15b1adda058c10e066d0c01b40ec950eb2911
Author: alexandre <alexandre.careil@supelec.fr>
Date:   Sun Feb 16 16:31:51 2014 +0100

    [CompteRendu] Modification de la page de garde

commit 4f1aa9118121e69652afc65548b2efc2445ff699
Author: alexandre <alexandre.careil@supelec.fr>
Date:   Sun Feb 16 16:31:24 2014 +0100

    [git] modification du .gitignore

commit 0f548ee172c523eaaa6c51e9db113096470fba3f
Author: ThiDiff <juanmanuel.munozperez@gmail.com>
Date:   Thu Feb 13 10:18:38 2014 +0100

    [GUI] Ajout de la classe Fenetre
\end{verbatim}


\par Cette commande affiche la liste des commits effectués sur le projet. Vous pouvez vous apercevoir que certains commits ne sont pas de vous.

\subsubsection{Synchronisation avec le projet de référence}
\label{sec:synchr-avec-le}

\par Une fois toutes vos modifications effectuées et bien enregistrées comme indiqué dans le paragraphe précédent, il faut les rendre accessibles à toutes les autres personnes qui travaillent également sur le projet. Il faut d'abord commencer par récupérer les éventuelles modifications qui auraient pu être ajoutées pendant votre travail. Pour cela on crée une branche \texttt{pull} sur laquelle on récupère le projet distant, puis on les fusionne avec les votres.

\par On crée la branche \texttt{pull}
\begin{verbatim}
> git checkout -b pull
\end{verbatim}

\par On récupère la dernière version du projet :
\begin{verbatim}
> git pull adresse_du_projet master
\end{verbatim}

\par Dans notre cas :
\begin{verbatim}
> git pull https://github.com/hhalex/PLS14 master
\end{verbatim}

\par Une fois fait, on retourne sur la branche \texttt{master} (la branche de base) et on fusionne le contenu de \texttt{pull} :

\begin{verbatim}
> git checkout master
> git merge pull
\end{verbatim}

\par Dans notre exemple :

\begin{verbatim}
Alex > git pull https://github.com/hhalex/PLS14 master
From https://github.com/hhalex/PLS14
 * branch            master     -> FETCH_HEAD
Already up-to-date.
Alex > git checkout master
Switched to branch 'master'
Your branch is ahead of 'origin/master' by 2 commits.
  (use "git push" to publish your local commits)
Alex > git merge pull
Already up-to-date.
\end{verbatim}

\par Maintenant que votre branche \texttt{master} est fin prête, on peut l'envoyer sur le projet distant. Pour cela, tapez :

\begin{verbatim}
> git push https://github.com/hhalex/PLS14 master
\end{verbatim}

\par Dans notre exemple :
\begin{verbatim}
 > git push https://github.com/hhalex/PLS14 master
Username for 'https://github.com': hhalex
Password for 'https://hhalex@github.com': ****

Counting objects: 10, done.
Delta compression using up to 4 threads.
Compressing objects: 100% (7/7), done.
Writing objects: 100% (7/7), 975 bytes | 0 bytes/s, done.
Total 7 (delta 4), reused 0 (delta 0)
To https://github.com/hhalex/PLS14
   0f548ee..afe15b1  master -> master
\end{verbatim}

\subsection{OAR : résumé des commandes utilisées}
\label{sec:oar-:-resume}

\subsection{Vocabulaire du High Performance Computing (HPC)}
\label{sec:vocab-spec-du}


%%% Local Variables: 
%%% mode: latex
%%% TeX-master: "CompteRendu"
%%% End: 
