
\section{Logiques de description}
\label{sec:logiq-de-descr}

\subsection{Première approche}
\label{sec:premiere-approche}

\emph{Cette partie est très largement basée sur `The Description Logic Handbook`}
\par La \emph{logique de description} est un terme récent désignant une famille de formalismes de représentation des connaissances, pour un domaine d'application donné. L'idée est de définir des concepts pertinents afin de caractériser les individus du domaine.

\begin{exemple}
  Par exemple, considérons les concepts \textsc{Homme}, \textsc{HommeBrun}, \textsc{Femme}, et l'individu \emph{Thomas}. On dit donc que \emph{Thomas} est une instance du concept \textsc{Homme}.
\end{exemple}
\par Contrairement à ses prédécesseurs, la logique de description jouit d'une sémantique basée sur la logique de premier ordre, ce qui la rend à la fois très mathématique, et surtout exploitable par des algorithmes de raisonnement. Il est en effet possible d'inférer dessus pour en déduire des résultats implicites, à partir des connaissances explicites de la base de connaissances.
\par Ces raisonnements tiennent compte de plusieurs relations entre concepts, dont celle consistant à les décliner précisément en sous-concepts. C'est l'inclusion(\emph{subsumption}) qui permet d'établir des relations sous-concepts/super-concepts. Pour des concepts d'une terminologie donnée, cela permet donc de les classifier, et ainsi déterminer si un individu est une instance d'un certain concept ou pas, ce qui apporte des informations supplémentaires. Cette classification permet de plus d'accélérer le raisonnement des algorithmes, chose non négligeable comme nous allons très vite le constater.
\par En effet, bien qu'ayant de multiples avantages, la logique de description nous confronte à des problèmes à la fois de décidabilité, et de complexité des algorithmes d'inférence, qui ont été l'objet d'importantes recherches dans le domaine. Les exigences vont même plus loin, une réponse donnée en un temps fini ne l'est pas forcément en un temps raisonnable. La complexité est très largement liée à l'expressivité du formalisme choisi pour concevoir la base de connaissances, il s'agira donc de faire un compromis satisfaisant. Les logiques de description pauvres, bien qu'étant efficaces pour l'utilisation de raisonnements, ne permettent pas de décrire autant de concepts d'un domaine donné que leurs homologues expressifs. Ces derniers souffrent d'une complexité très élevée lors de l'execution de raisonnements, et même parfois de problèmes d'indécidabilité, mais permettent une description plus riche du domaine choisi.
\par Les logiques de description héritent de certaines idées autour desquelles étaient bâtis les \emph{réseaux sémantiques} :
\begin{itemize}
\item Les briques élémentaires de la syntaxe sont les concepts atomiques (prédicats unaires), les rôles atomiques (prédicats binaires), et les individus (constantes).
\item L'expressivité du langage est restreinte en cela qu'un nombre limité de constructeurs est utilisé pour élaborer des concepts complexes.
\item Des connaissances implicites peuvent être déduites automatiquement grâce à des procédures d'inférence. Là où les \emph{réseaux sémantiques} utilisaient des relations \texttt{est-un}, créées par l'utilisateur pour simuler des relations d'inclusion, la logique de description permet de l'employer nativement.
\end{itemize}


\subsection{Définition du formalisme de base}
\label{sec:defin-du-form}

\par Une base de connaissances est constituée de deux composants, la \emph{ABox} et la \emph{TBox}.

\definition{TBox, Terminologie, concept, relation binaire}
{C'est la partie de la base de connaissances qui contient le vocabulaire du domaine d'application. Cette \emph{terminologie} est constituée d'une suite de définitions de concepts, et de rôles. Chaque concept désigne un ensemble d'individus, lesquels seraient alors des instances de ce concept.}

\par Dans une base de connaissances, beaucoup de concepts utilisent d'autres concepts déjà définis. Certains d'entre eux ne sont mêmes définis qu'à partir de simples requêtes.

\definition{ABox}
{La ABox (\emph{assertion Box}) est la deuxième partie de la base de connaissances. Elle permet d'établir des assertions sur les \textsc{individus}, grâce à la \emph{TBox}.}

\par La syntaxe employée pour définir des concepts dépend de la logique de description utilisée, mais il est toujours possible de revenir à la logique de premier ordre, ou du moins à une version légèrement étendue dans certains cas.

\par En plus de ces deux éléments, la base de connaissances met à disposition des services d'inférence, nous permettant de savoir si ce qui est stocké a un sens, puisqu'il ne s'agit pas de données brutes, mais bien d'assertions. C'est pourquoi le principal enjeu sera de savoir si une \emph{ABox} est cohérente, pour déterminer si la base de connaissances a un sens ou pas.

\par Afin d'éviter toute assertion susceptible de provoquer un non-sens, les \emph{règles} permettent d'introduire certaines restrictions dans leur définition.

\begin{exemple}
  Un homme A peut être le fils d'un autre homme B, mais alors l'homme B ne peut
  être le fils de l'homme A. La relation \texttt{est-fils} ne peut donc être
  symétrique.
\end{exemple}

\subsubsection{Les langages de description}
\label{sec:les-langages-de}

\par Comme nous l'avons vu, les langages de description se distinguent par leur syntaxe plus ou moins riche, permettant une inférence plus ou moins rapide.

\par Commençons par définir des notations utiles pour la suite :
\begin{itemize}
\item A,B : Concepts atomiques
\item R : Pour les relations binaires
\item C,D : Concepts descriptifs (concepts construits à partir d'autres concepts)
\end{itemize}

\paragraph{$\mathcal{AL}$ : le langage de description de base}
\label{sec::le-langage-de}

\par Pour former des concepts descriptifs (et non atomiques) avec le langage $\mathcal{AL}$, on utilise les règles syntaxiques suivantes :
\begin{tabbing}
  C, D  $\longrightarrow{}$ \= A | \ \ \ \ \ \ \ \= (concept atomique)\\
  \> $\top{}$ | \>(le concept universel : tout concept en est un sous-concept)\\
  \> $\bot{}$ | \>(bottom concept : sous-concept de tout concept) \\
  \> $\neg{}A$ | \>(négation atomique) \\
  \> $C\sqcap D$ | \>(intersection) \\
  \> $\forall{}R.C$ | \> (restriction de valeur) \\
  \> $\exists R.\bot$ | \> (quantificateur d'existence limité : on ne peut définir le concept en relation)\\

\end{tabbing}
\definition{$\mathcal{FL}^-$}{Langage $\mathcal{AL}$ dépourvu de la négation}  
\definition{$\mathcal{FL}_0$}{Langage $\mathcal{AL}$ dépourvu de la négation et du quantificateur existentiel. Autrement dit, c'est le langage $\mathcal{FL}^-$ sans le quantificateur existentiel.}

% exemples

% 

% 

\par La définition des concepts  donnée ci-dessus ne favorise pas leur manipulation à l'aide d'algorithmes, c'est pourquoi nous allons leur associer une sémantique formelle. Nous parlerons ainsi d'interprétation $\mathcal{I}$.

\definition{Interprétation}
{Une interprétation $\mathcal{I}$ est la donnée d'un ensemble non vide $\Delta^\mathcal{I}$ (le domaine de l'interprétation) et une fonction d'interprétation à qui tout concept $A$ associe l'ensemble $A^\mathcal{I} \subseteq \Delta^\mathcal{I}$, et à qui tout rôle atomique $R$ associe une relation binaire $R^\mathcal{I} \subseteq \Delta^\mathcal{I} \times \Delta^\mathcal{I}$}

\par Cette définition ne s'appliquant qu'aux briques élémentaires du langage, on l'étend aux concepts descriptifs grâce aux définitions inductives suivantes :

% def inductives

\begin{align}
  \top{}^\mathcal{I} & = \Delta^\mathcal{I} \\
  \bot{}^\mathcal{I} & = \emptyset \\
  (\neg{}A)^\mathcal{I} & = \Delta^\mathcal{I}\backslash A^\mathcal{I}\\
  (C \sqcap D)^\mathcal{I} & = C^\mathcal{I} \cap D^\mathcal{I} \\
  ( \forall R.C )^\mathcal{I} & = \{ a \in \Delta^\mathcal{I} / \forall b.(a,b) \in R^\mathcal{I} \longrightarrow b \in C^\mathcal{I} \} \\
  ( \exists R.\bot )^\mathcal{I} & = \{ a \in \Delta^\mathcal{I} / \exists b.(a,b) \in R^\mathcal{I} \}
\end{align}

\definition{Concepts équivalents}
{Deux concepts C et D  sont équivalents \emph{ssi} $C\equiv D$ \emph{ssi} $C^\mathcal{I} = D^\mathcal{I}$ pour toute interprétation $\mathcal{I}$. }

\begin{exemple}
  $\forall hasChild.Female \sqcap \forall hasChild.Student \equiv \forall hasChild.(Female \sqcap Student)$
\end{exemple}

\par Dans l'exemple ci-dessus, on a simplement \emph{factorisé} l'expression.

\paragraph{La famille des langages $\mathcal{AL}$}
\label{sec:la-famille-des}

\par On peut ajouter d'autres constructeurs afin de rendre le langage plus expressif. Nous verrons ensuite que certains d'entre eux peuvent être définis à partir de ceux rencontrés précédemment.

\par L'union de concepts, notée $C \sqcup D$ est interprétée de la manière suivante :
\begin{displaymath}
  \textcolor{red}{[\mathcal{U}]} : (C \sqcup D)^\mathcal{I} = C^\mathcal{I} \cup D^\mathcal{I}
\end{displaymath}

\par Le quantificateur existentiel complet permet d'outrepasser la restriction imposée par le quantificateur existentiel limité. Là où $\exists R.\bot$ ne permet que de désigner les concepts participant à une relation R, le quantificateur complet de spécifier le concept avec lequel il est en relation.
\begin{displaymath}
  \exists R.\bot = \{C | \exists X concept / CRX\}
\end{displaymath}

\par L'interprétation est donc pratiquement la même que celle du quantificateur existentiel limité, à ceci près qu'on spécifie la nature de b.

\begin{displaymath}
  \textcolor{red}{[\mathcal{E}]} : ( \exists R.C )^\mathcal{I}  = \{ a \in \Delta^\mathcal{I} / \exists b.(a,b) \in R^\mathcal{I} \wedge b \in C^\mathcal{I} \}
\end{displaymath}

\par La restriction numérique permet de spécifier un nombre minimal ou maximal de relations auxquelles participe un concept. Par exemple, un homme peut avoir un frère, et on peut vouloir les hommes qui ont au moins deux frères.

\begin{displaymath}
  \textcolor{red}{[\mathcal{N}]} : ( \geq nR )^\mathcal{I}  = \left\{ a \in \Delta^\mathcal{I} | |\{b| (a,b) \in R^\mathcal{I}\}| \geq n \right\}
\end{displaymath}

\begin{displaymath}
  \textcolor{red}{[\mathcal{N}]} : ( \leq nR )^\mathcal{I}  = \left\{ a \in \Delta^\mathcal{I} | |\{b| (a,b) \in R^\mathcal{I}\}| \leq n \right\}
\end{displaymath}

\par Comme prévu, certains opérateurs sont redondants, les  activer tous n'est donc pas nécessaire. En effet, la négation peut être utilisée pour construire les opérateurs d'union et de quantification existentielle complète, et réciproquement, l'utilisation de ces deux derniers opérateurs permet d'exprimer la négation. Ainsi, on constate l'équivalence suivante :

\begin{align}
  \mathcal{ALC}  & \equiv \mathcal{ALUE}\\
  \mathcal{ALCN}  & \equiv \mathcal{ALUEN}
\end{align}

\important{Tous les langages $\mathcal{AL}$ peuvent être utilisés avec les seules lettres $\mathcal{U}$, $\mathcal{E}$ et $\mathcal{N}$}

\paragraph{Les langages de description en tant que fragments de la logique de prédicats}
\label{sec:les-langages-de-1}

\par Le formalisme déjà établi peut être amélioré. Jusqu'alors, une interprétation $\mathcal{I}$ permettait d'associer respectivement à un concept et à un rôle  une relation unaire et une relation binaire sur $\Delta^\mathcal{I}$. Des prédicats peuvent également leur être associés.
\begin{exemple}
  Considérons le concept $C$, exprimons son interprétation par $\mathcal{I}$: 
  \begin{displaymath}
    C^\mathcal{I} = \left\{ x \in \Delta^\mathcal{I} | \phi_C(x) \right\}
  \end{displaymath}
\end{exemple}

\important{Si pour toute interprétation $\mathcal{I}$, l'ensemble des $x$ de $\Delta^\mathcal{I}$ tels que $\phi_C(x)$ est exactement $C^\mathcal{I}$, alors $\phi_C$ est le prédicat logique associé à $C$.}

\par On associe donc à tout concept $A$ sa formule logique $A(x)$. Les constructeurs sont donc transformés de la manière suivante :

\begin{align}
  \sqcap & \longrightarrow \wedge \\
  \sqcup & \longrightarrow \vee \\
  \neg & \longrightarrow \neg
\end{align}

\par Afin de pouvoir procéder récursivement sur les expressions, on utilise les expressions suivantes :

\begin{align}
  \phi_{\exists R.C}(y) & = \exists x.R(y,x) \wedge \phi_C(x) \\
  \phi_{\forall R.C}(y) & = \forall x.R(y,x) \longrightarrow \phi_C(x)\\
  \phi_{\geq nR}(x) & = \exists y_1, \ldots, y_n.R(x,y_1)\wedge \ldots \wedge R(x,y_n) \wedge \bigwedge_{i<j}y_i \neq y_j \label{label_geq}\\
  \phi_{\leq nR}(x) & = \forall y_1, \ldots, y_{n+1}.R(x,y_1)\wedge \ldots \wedge R(x,y_{n+1}) \longrightarrow \bigvee_{i<j}y_i = y_j \label{label_leq}
\end{align}

\par L'équation \ref{label_geq} traduit bien qu'on veut au moins $n$ éléments $y_i$ en relation $R$ avec $x$. L'équation \ref{label_leq} permet de sélectionner les éléments x qui sont en relation via $R$ avec au plus $n$ éléments. Ainsi pour un $x$ membre de cet ensemble, pour n'importe quel $(n+1)-uplet$ d'éléments $y_i$ en relation avec $x$, il existe deux éléments de ce $(n+1)-uplet$ qui sont identiques.

\par Il est à noter que l'utilisation de ces restrictions numériques entraîne l'utilisation du symbole $=$. Les concepts exprimés sans restriction numérique n'en ont pas besoin.

\par La syntaxe ainsi établie est plus concise, et se prête aisément aux algorithmes.

\subsubsection{Terminologies}
\label{sec:terminologies}

\par Une terminologie est précisément un ensemble d'axiomes dits terminologiques. Parmi eux, nous distinguerons les définitions de concepts. Selon certaines définitions, qui pourront parfois être cycliques, nous classifierons les terminologies en plusieurs catégories afin de savoir quels théorèmes peuvent leur être appliqués.

\paragraph{Axiomes terminologiques}
\label{sec:axiom-term}

\definition{Axiome terminologique}{
  Expression de la forme $C\sqsubseteq D$ ou $C\equiv D$ pour les concepts, et $R\sqsubseteq S$ ou $R\equiv S$ pour les rôles.
}
\par Une interprétation $\mathcal{I}$ satisfait l'axiome :
\begin{itemize}
\item $C\sqsubseteq D$ \emph{ssi} $C^{\mathcal{I}} \subseteq D^{\mathcal{I}}$
\item $C\equiv D$ \emph{ssi} $C^{\mathcal{I}} = D^{\mathcal{I}}$
\end{itemize}

\important{
  Soit une terminologie $\mathcal{T}$, et une interprétation $\mathcal{I}$. On dit que $\mathcal{I}$ satisfait $\mathcal{T}$ \emph{ssi} $\mathcal{I}$ satisfait chaque axiome de $\mathcal{T}$.
}

\definition{Modèle d'une terminologie}{
  $\mathcal{I}$ est un modèle de $\mathcal{T}$ \emph{ssi} $\mathcal{I}$ satisfait $\mathcal{T}$.
}

\important{
  Deux terminologies $\mathcal{T}$ et $\mathcal{T\prime}$ sont dites équivalentes \emph{ssi} elles ont les mêmes modèles.
}

\paragraph{Définitions}
\label{sec:definitions}

\par Une définition est utilisée pour définir un nom symbolique, afin de désigner de manière explicite un concept complexe (souvent long à lire, et parfois fastidieux à comprendre).

\par L'exemple suivant permet de comprendre facilement comment on peut définir plein de concepts dits complexes à partir de quelques concepts et rôles atomiques.

\begin{exemple}
  \begin{align}
    \textsc{Homme} & \equiv (\neg \textsc{Femme})\sqcap\textsc{Personne} \\
    \textsc{Mère} & \equiv (\textsc{Femme})\sqcap\exists aEnfant.\textsc{Personne}
  \end{align}
\end{exemple}

\par Cela dit, nous pouvons préciser la définition de terminologie déjà donnée :

\definition{Terminologie}{
  Une terminologie est un ensemble fini de définitions uniques, c'est-à-dire qu'aucun nom symbolique n'est défini plus d'une fois.
}

\par On définit deux ensembles importants pour une \emph{TBox} :
\begin{align}
  \mathcal{N}_{\mathcal{T}} & =\mbox{ \{ noms symboliques\} = "membre gauche dans les axiomes" = concepts définis} \\
  \mathcal{B}_{\mathcal{T}} & =\mbox{ \{ symboles de base\} = "concepts utilisés à droite dans les  axiomes" = concepts primitifs}
\end{align}

\definition {Interprétation de base}{
  Soit $\mathcal{T}$ une terminologie. $\mathcal{J}$ est une interprétation de base de $\mathcal{T}$ \emph{ssi} elle n'interprète que les concepts primitifs de $\mathcal{T}$.
}

\definition{Interprétation étendue}{
  Soit $\mathcal{J}$ une interprétation de base. Une interprétation $\mathcal{I}$ qui interprète aussi les concepts définis est une extension de $\mathcal{J}$ \emph{ssi}
  \begin{itemize}
  \item $\Delta^{\mathcal{I}} = \Delta^{\mathcal{J}}$
  \item il y a concordance entre les symboles de base de $\mathcal{I}$ et $\mathcal{J}$.
  \end{itemize}
  
}


\definition{Terminologie définitoriale}{
  Une terminologie $\mathcal{T}$ est définitoriale \emph{ssi} toute interprétation de base a exactement une extension qui est un modèle de $\mathcal{T}$.
}

\par On peut voir une terminologie définitoriale comme une terminologie  bien déterminée, c'est-à-dire qu'on sait précisément à quoi servent les concepts primitifs. Ce n'est pas toujours évident, notamment avec des définitions cycliques pouvant donner lieu à plusieurs interprétations possibles. Dans ce cas, la terminologie n'est pas définitoriale puisqu'elle peut être interprétée de différentes manières.

\par L'exemple suivant permet d'illustrer les possibiliés d'interprétations multiples.

\begin{exemple}
  \begin{center}
    \textsc{Humain} $\equiv$ \textsc{Animal} $\sqcap$ $\forall$ aParent.\textsc{Humain}
  \end{center}
\end{exemple}

\par La définition ci-dessus peut désigner toute espèce animale, il y a donc beaucoup d'interprétations possibles. C'est cette absence d'unicité qui enlève son caractère définitorial à cette terminologie.

\important{
  Soit $\mathcal{T}$ une terminologie définitoriale, alors toute terminologie équivalente à $\mathcal{T}$ est aussi définitoriale.
}

\definition{Utilisation, utilisation directe}{
  Soient $A$ et $B$ deux concepts atomiques de $\mathcal{T}$.
  \begin{itemize}
  \item $A$ \emph{utilise directement} $B$ dans $\mathcal{T}$ \emph{ssi} $B$ apparaît dans le membre droit de l'axiome définissant $A$.
  \item $A$ \emph{utilise} $B$ dans $\mathcal{T}$ \emph{ssi} $B$ apparaît dans la fermeture transitive de la relation \emph{utilise directement}.
  \end{itemize}
}

\important{
  \emph{utilise directement} $\Rightarrow$ \emph{utilise}
}

\par Il paraît donc naturel de dire qu'une terminologie $\mathcal{T}$ contient un cycle \emph{ssi} un des concepts définis dans $\mathcal{T}$ s'utilise lui-même.

\par Comme nous l'avons vu, les terminologies cycliques peuvent ne pas être définitoriales, on ne peut rien dire a priori. Pour ce qui est des terminologies acycliques, on peut par contre affirmer qu'elles sont toutes définitoriales.

\lemme{termacy_def}{
  Toute terminologie acyclique est définitoriale
}
\par La raison est simple, si on procède par équivalences successives en remplaçant les noms symboliques par leurs expressions en fonction des concepts primitifs, on obtient des concepts exprimés uniquement à l'aide de symboles de base. Ce procédé d'expansion peut cependant produire une terminologie de taille exponentielle en celle de l'originale.

\lemme{term_acy}{
  Soient $\mathcal{T}$ une terminologie acyclique et $\mathcal{T}^\prime$ son expansion. \\
  \begin{itemize}
  \item $\mathcal{T}$ et $\mathcal{T}^\prime$ ont même symboles de base, et même noms symboliques.
  \item $\mathcal{T}$ et $\mathcal{T}^\prime$ sont équivalentes.
  \item $\mathcal{T}$ et $\mathcal{T}^\prime$ sont définitoriales.
  \end{itemize}
}

\par Comme souligné précédemment, lorsqu'une terminologie est cyclique, on ne peut rien dire. Dans certains cas, elles sont quand même définitoriales.

\begin{exemple}
  \begin{displaymath}
    A\mbox{ } \equiv \mbox{ }\forall R.B \sqcup \exists R.(A\sqcap \neg A)
    \mbox{ mais }A \sqcap \neg A \mbox{ } \equiv \mbox{ } \bot \Rightarrow A \mbox{ }\equiv\mbox{ } \forall R.B
  \end{displaymath}
\end{exemple}

\par Il vient ensuite un théorème qui n'est rien d'autre qu'une reformulation d'un théorème établi par Beth en 1972.

\theoreme{Equivalence acyclique}{
  Toute terminologie $\mathcal{ALC}$ définitoriale est équivalente à une
  terminologie acyclique.  
}

\paragraph{Sémantique des points fixes pour les cycles terminologiques}
\label{sec:semant-des-points}

\par Plutôt que de vouloir se débarrasser d'une syntaxe cyclique, parfois beaucoup plus intuitive et adaptée que les notations acycliques, on introduit une sémantique dite des points fixes, permettant justement de traiter au mieux ces définitions.

\par Introduisons donc cette nouvelle sémantique. Nous considérons désormais les terminologies non plus comme des ensembles, mais comme des fonctions à qui chaque concept nommé renvoie son expression. En clair, une terminologie associe à un membre de gauche son membre de droite associé.
\begin{exemple}
  \par Si nous avons la terminologie $\mathcal{T}$ suivante :
  \begin{align}
    \textsc{Homme} & \equiv (\neg \textsc{Femme})\sqcap\textsc{Personne} \\
    \textsc{Mère} & \equiv (\textsc{Femme})\sqcap\exists aEnfant.\textsc{Personne}
  \end{align}
  \begin{align}
    \mbox{alors }
    \mathcal{T}(\textsc{Homme}) & = (\neg \textsc{Femme}) \sqcap \textsc{Personne} \\
    \mathcal{T}(\textsc{Mère}) & = (\textsc{Femme})\sqcap\exists aEnfant.\textsc{Personne}
  \end{align}
\end{exemple}

\par Avec ces notations, une interprétation $\mathcal{I}$ est un modèle de $\mathcal{T}$ \emph{ssi} $A^\mathcal{I} = \mathcal{T}(A)^\mathcal{I}$ pour tout nom symbolique $A$ de $\mathcal{T}$.

\par Allons plus loin. Soit $\mathcal{T}$ une terminologie, et $\mathcal{J}$ une interprétation de base. Notons $Ext_\mathcal{J}$ l'ensemble des extensions de $\mathcal{J}$, et $\mathcal{T}_\mathcal{J} : Ext_\mathcal{J} \longrightarrow Ext_\mathcal{J}$ l'application à qui toute interprétation étendue $\mathcal{I}$ associe $\mathcal{T}(\mathcal{J})$, l'interprétation  définie par $A^{\mathcal{T}(\mathcal{J})} = \mathcal{T}(A)^\mathcal{I}$ pour tout nom symbolique $A$.

\lemme{modele_pointfixe}{
  Soient  $\mathcal{T}$ une terminologie, $\mathcal{I}$ une interprétation, et $\mathcal{J}$ l'interprétation $\mathcal{I}$ restreinte aux symboles de base de $\mathcal{T}$. $\mathcal{I}$ est un modèle de $\mathcal{T}$ \emph{ssi} $\mathcal{I}$ est un point fixe de $\mathcal{T}_\mathcal{J}$.
}

\par On note que par conséquent, une terminologie est définitoriale \emph{ssi} toutes ses interprétations de base ont une unique extension qui est un point fixe de $\mathcal{T}_\mathcal{J}$.

\par Mais évidemment, la présence d'un unique point fixe n'est pas systématique, c'est pourquoi il est nécessaire d'introduire les notions suivantes, nous permettant d'appliquer des résultats aux terminologies qui ne sont pas concernées par les résultats précédents.

\definition{Relation d'ordre partiel sur l'ensemble des interprétations}{
  Soient $\mathcal{I}$ et $\mathcal{I}\prime$ deux interprétations de $\mathcal{T}$. On note $\mathcal{I} \preceq \mathcal{I}^\prime$ \emph{ssi} $\forall A$ nom symbolique de $\mathcal{T}$,  $A^\mathcal{I} \subseteq A^{\mathcal{I}^\prime}$
}

\definition{Least Fixpoint (lfp), Greatest Fixpoint (gfp)}{
  Soit $\mathcal{I}$ un point fixe de $\mathcal{T}_\mathcal{J}$. $\mathcal{I}$ est un \emph{Least Fixpoint} (respectivement \emph{Greatest Fixpoint}) de $\mathcal{T}$ \emph{ssi} $\forall \mathcal{I}^\prime$ point fixe de $\mathcal{T}$, $\mathcal{I} \preceq \mathcal{I}^\prime$ (respectivement $\mathcal{I} \succeq \mathcal{I}^\prime$ ).
}

\paragraph{Existence de modèles points fixes}
\label{sec:existence-de-modeles}

\par LFP et GFP n'existent pas nécessairement dans toute terminologie.

\begin{exemple}
  \begin{displaymath}
    A \equiv \forall. \not A
  \end{displaymath}
  \par Notons $\Delta^\mathcal{J} = {(a,b)}$ et $R^\mathcal{J} = {(a,b),(b,a)}$, pour $\mathcal{J}$ l'interprétation de base de la terminologie précédente.
  \par Les deux interprétations étendues $\mathcal{I}_1$ et $\mathcal{I}_2$ définies par $A^{\mathcal{I}_1} = {a}$ et $A^{\mathcal{I}_2} = {b}$ ne sont pas comparables au sens de la relation d'ordre partielle "$\preceq$"
\end{exemple}

\par Nous allons donc identifier les terminologies qui possèdent un LFP et GFP. Pour cela on se base sur des résultats déjà établis de la théorie du treillis. Gardons en mémoire qu'un treillis est complet si toute famille d'éléments a un plus petit majorant.

\par On montre que $(Ext_\mathcal{J}, \preceq)$ est un treillis complet.

\theoreme{Théorème de Tarski (Point fixe)}
{
  Une fonction monotone sur un treillis complet, l'ensemble des points fixes est non vide et constitue un treillis complet.
}

\definition{Terminologie monotone}{Une terminologie $\mathcal{T}$ est monotone si sa fonction associée $\mathcal{T}_\mathcal{J}$ est monotone.}

\important{Une terminologie $\mathcal{T}$ monotone vérifie le théorème de Tarsky, et possède donc un LFP et un GFP.}

\par Nous tentons donc désormais d'identifier les terminologies monotones.
\par Un critère simple permet de les trouver : Une $\mathcal{ALCN}-$terminologie sans négation est monotone.

\par On en déduit le lemme suivant :

\lemme{negation-free}{
  Si $\mathcal{T}$ est une terminologie sans négation et $\mathcal{J}$ une interprétation de base, alors il existe des extensions de $\mathcal{J}$ qui sont des modèles LFP et GFP de $\mathcal{T}$.
}

\par On peut même donner un résultat plus fort en disant que si le graphe de la terminologie ne présente que des cycles avec un nombre pair d'arcs négatifs, alors la terminologie est monotone (syntaxiquement monotone). Le graphe est fait de telle manière que tout nom symbolique $A$ est relié par des arcs aux concepts utilisés pour sa définition. Pour une terminologie cyclique on obtient donc des cycles dans le graphe.

\paragraph{Terminologies avec axiomes d'inclusion}
\label{sec:term-avec-axiom}

\par Certains concepts sont difficilement définissables complètement. Pourtant dans ces cas là, on est capable de lister quelques conditions nécessaires pour décrire au minimum les concepts en question. Une spécialisation est une inclusion utilisée pour restreindre un concept atomique. L'utilisation des spécialisations enlève son caractère définitorial à une ontologie même si celle ci est acyclique.
\par Nous allons donc nous débarrasser formellement de ces spécialisations en normalisant notre terminologie.

\begin{exemple}
  \begin{displaymath}
    Woman \sqsubseteq Person
  \end{displaymath}

  Nous normalisons ainsi le concept spécialisé :
  \begin{displaymath}
    Woman \sqcap \bar{Woman} \sqcap Person
  \end{displaymath}
\end{exemple}

\lemme{normalisation}{
  Soit  $\mathcal{T}$ une terminologie généralisée et $\bar{\mathcal{T}}$ sa normalisation.
  \begin{itemize}
  \item Tout modèle de $\bar{\mathcal{T}}$ est un modèle de $\mathcal{T}$.
  \item Pour tout modèle $\mathcal{I}$ de $\mathcal{T}$, il y a un modèle $\bar{\mathcal{I}}$ de $\bar{\mathcal{T}}$ qui a le même domaine que $\mathcal{I}$ et est définie pour les mêmes noms symboliques et relations de $\mathcal{T}$ que $\mathcal{I}$.
  \end{itemize}
}

\par Les inclusions n'améliorent pas l'expressivité du langage, elles sont juste parfois plus pratiques à utiliser, même s'il faut essayer de les éviter...

\subsubsection{Descriptions de domaines}
\label{sec:descr-de-doma}

\paragraph{Assertions sur les individus}
\label{sec:assertions-sur-les}

\par Grâce au vocabulaire défini dans la \emph{TBox}, on définit les individus de la \emph{ABox}. Grâce aux concepts et aux rôles, on qualifie les individus:
\begin{displaymath}
  C(a), \mbox{ } R(b,c)
\end{displaymath}

\begin{exemple}
  Homme(Peter), Homme(Stéphane), hasChild(Peter, Stéphane)
\end{exemple}

\par On étend les interprétations aux individus de la \emph{ABox}. De cette manière, un individu $a$ peut être interprété par $a^\mathcal{I} \in \Delta^\mathcal{I}$. Il est très important de s'assurer que les noms d'individus, s'ils sont différents, désignent des instances différentes (Unique Name Assumption), et l'interprétation doit respecter cela. Donc si a et b sont des noms d'individus distincts, $a^\mathcal{I}\neq b^\mathcal{I}$.
\par Une interprétation $\mathcal{I}$ satisfait une assertion $C(a)$ \emph{ssi} $a^\mathcal{I} \in C^\mathcal{I}$. De même, $R(a,b)$ est satisfaite \emph{ssi} $(a^\mathcal{I},b^\mathcal{I}) \in R^\mathcal{I}$.

\paragraph{Noms d'individus dans les langages de description}
\label{sec:noms-dindividus-dans}

\par Il peut être parfois pratique d'autoriser l'utilisation de noms d'individus dans le langage de description (pas seulement dans la \emph{ABox}).

\par Par exemple un concept peut être décrit de la manière suivante :
\begin{displaymath}
  \{a_1,\dots,a_n\}^\mathcal{I} = \{a_1^\mathcal{I},\dots, a_n^\mathcal{I}\}
\end{displaymath}

\par On en déduit que l'union $\sqcup$, en complément d'un constructeur de singleton $\{a\}$ permet de décrire avec suffisamment d'expressivité des ensembles finis. Voyez plutôt, les concepts suivants sont équivalents :
\begin{displaymath}
  \{a_1,\dots, a_n\} \equiv \{a_1\}\sqcup\dots \sqcup\{a_n\}
\end{displaymath}

\par Pour les relations, la notation avec individus peut permettre de désigner un autre ensemble d'individus:

\begin{displaymath}
  (R:a)^\mathcal{I} = \{d\in \Delta^\mathcal{I}|(d,a^\mathcal{I} \in R^\mathcal{I}\}
\end{displaymath}

\par L'ensemble des individus vérifiant la relation R avec a (tels que dRa)

\subsection{Inférences}
\label{sec:inferences}

\par Le problème est avant tout de déterminer si la \emph{ABox} est cohérente ou non.

\subsubsection{Taches de raisonnement pour les concepts}
\label{sec:tach-de-rais}

\par Le but est ici de déterminer si tous les concepts définis dans la \emph{TBox} peuvent être satisfaits, c'est la base de l'inférence. En plus de vouloir savoir si un concept peut être satisfait, il peut être également intéressant de savoir s'il est subsumé par un autre concept.

\begin{itemize}
\item Satisfiabilité : Un concept $C$ est satisfiable au sens de $\mathcal{T}$ s'il existe un modèle $\mathcal{I}$ de $\mathcal{T}$ tel que $C^\mathcal{I}$ est non vide.
\item Subsumption : Un concept $C$ est subsumé par un concept $D$ au sens de $\mathcal{T}$ si $C^\mathcal{I} \subseteq D^\mathcal{I}$. On écrit dans ce cas : $C \subseteq_\mathcal{T} D$ ou $\mathcal{T} \models C \subseteq D$.
\item Equivalence : Deux concepts C et D sont dits équivalents  au sens de $\mathcal{T}$ si $C^\mathcal{I} = D^\mathcal{I}$ pour tout modèle $\mathcal{I}$ de $\mathcal{T}$. Dans ce cas, on note $C \equiv_\mathcal{T} D$ ou $\mathcal{T} \models C \equiv D$.
\item Concepts disjoints : Deux concepts C et D sont disjoints au sens de $\mathcal{T}$ si $C^\mathcal{I} \cap D^\mathcal{I} = \emptyset$ pour tout modèle $\mathcal{I}$ de $\mathcal{T}$.
\end{itemize}

\par Le principe de base des mécanismes d'inférence se base sur la subsumption des concepts. Cela permet d'ailleurs d'effectuer d'autres types d'inférences comme nous allons le constater avec les équivalences suivantes.

\theoreme{Réduction à la subsumption}{
  Pour des concepts C et D, nous avons :
  \begin{itemize}
  \item C insatisfiable $\Longleftrightarrow{}$ C est subsumé par $\bot$
  \item C et D sont équivalents $\Longleftrightarrow{}$ C est subsumé par D et D est subsumé par C
  \item C et D sont disjoints $\Longleftrightarrow{}$ $C\cap D$ est subsumé par $\bot$
  \end{itemize}

  Tout cela ne s'appliquant qu'au sens d'une \emph{TBox} T
}

\par Si le langage permet d'utiliser la négation, on peut réduire les problèmes à un problème de satisfiabilité grâce aux équivalences suivantes :

\theoreme{Réduction à l'insatisfiabilité}{
  Pour des concepts C et D, nous avons :
  \begin{itemize}
  \item C est subsumé par D $\Longleftrightarrow{}$ $C\cap\neg D$ est insatisfiable
  \item C et D sont équivalents $\Longleftrightarrow{}$ $C\cap\neg D$ et $\neg C\cap D$ sont insatisfiables
  \item C et D sont disjoints $\Longleftrightarrow{}$ $C\cap D$ est insatisfiable
  \end{itemize}
  
  Tout cela ne s'appliquant qu'au sens d'une \emph{TBox} T
}

\par Bien sûr les dernières équivalences ne sont applicables que si le langage considéré possède la négation totale. SI tel n'était pas le cas, la réduction serait d'une nature différente, et par conséquent la complexité de l'inférence serait également différente.

\par L'exemple suivant se charge d'illustrer d'un exemple d'insatisfiabilité chacune des relations vues :

\theoreme{Réduction d'insatisfiabilité}{
  \par Soit C un concept, alors les assertions suivantes sont équivalentes
  \begin{itemize}
  \item C est insatisfiable
  \item C est subsumé par $\bot$
  \item C et $\bot$ sont équivalents
  \item C et $\top$ sont disjoints
  \end{itemize}
}


%%% Local Variables: 
%%% mode: latex
%%% TeX-master: "CompteRendu"
%%% End: 
