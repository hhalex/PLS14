
\section{Plan de test}
\label{sec:plan-de-test}


\par L'application devra remplir certains tests, dont voici la liste :

\subsection{Affichage}
\label{sec:affichage}

\begin{table}[h!]
  \centering
  \begin{tabular}[h!]{|c|c|p{10cm}|}
    \hline \\
    1 & Boîte connexion MF & La boîte de connexion à la machine frontale doit s'afficher au lancement du programme. \\
    2 & Boîte connexion MF & La boîte de connexion à la machine frontale doit disparaître si on appuie sur le bouton connexion, avec un login et un mot de passe valides.\\
    3 & Boîte OAR & La boîte OAR doit s'afficher après une connexion avec succès à l'aide de la boîte de connexion MF. \\
    4 & Boîte OAR & La boîte OAR doit afficher deux champs de saisie, 3 boutons, et une zone de texte à hauteur variable. \\
    5 & Boîte OAR & La zone de texte de la boîte OAR doit afficher initialement le message d'accueil du shell distant auquel l'interface est connectée. \\

    \hline    
  \end{tabular}
  \caption{Liste des tests d'affichage à remplir}
  \label{tab:tests_affichage}
\end{table}

\subsection{Connexion}
\label{sec:connexion}

\begin{table}[h!]
  \centering
  \begin{tabular}[h!]{|c|c|p{10cm}|}
\hline \\
    10 & connexion à Ghome & affiche dans la console "Connecté à ghome.metz.supelec.fr sous le port 22"\\
    11 & connexion à Term2 & affiche dans la console "Connecté à term2.metz.supelec.fr sous le port xx"\\
\hline
  \end{tabular}
  \caption{Liste des tests de connexion à remplir}
  \label{tab:tests_connexion}
\end{table}

\subsection{Fonctionnalités}
\label{sec:fonctions}

\begin{table}[h!]
  \centering
  \begin{tabular}[h!]{|c|p{3cm}|p{10cm}|}
\hline \\
    15 & Bouton Connexion & avec un login et un mot de passe corrects : fait disparaître la fenêtre de connexion et affiche la fenêtre d'allocation des noeuds.\\
    16 & bouton "Allouer des noeuds" & affiche dans la zone de texte la réponse du shell distant à la commande "oarsub" envoyée grâce au clic sur le bouton.\\
    17 & bouton "Tuer le job courant" & affiche dans la zone de texte la réponse du shell distant à la commande "oardel" envoyée grâce au clic sur le bouton.\\
    18 & bouton "Infos le job courant" & affiche dans la zone de texte des informations relatives au job en cours d'execution.\\
\hline
  \end{tabular}
  \caption{Liste des tests de connexion à remplir}
  \label{tab:tests_connexion}
\end{table}

%%% Local Variables: 
%%% mode: latex
%%% TeX-master: "CompteRendu"
%%% End: 
