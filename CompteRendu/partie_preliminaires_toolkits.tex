\vskip 0.5cm

\begin{minipage}{0.30\linewidth}
\centering
  \includegraphics[width=4cm]{images/logos/genmymodel.png}
\end{minipage}\hfill
\begin{minipage}{0.65\linewidth}
\par Outre la plateforme Github, nous avons été amenés à travailler sur la plateforme collaborative genmymodel.com pour l'édition des diagrammes UML. La facilité de prise en main ainsi que la possibilité de se connecter via nos comptes Github ont été deux éléments déterminants dans le choix de celle-ci.
\end{minipage}

\vskip 0.5cm

\begin{minipage}{0.65\linewidth}
\par Au-delà des diagrammes UML, la richesse schématique de notre rapport est dûe à l'application web draw.io. Encore une fois, facilité et rapidité de prise en main ont été décisifs dans le choix de cette application. De plus, le large panel d'icônes offerts par cette dernière nous a permis de réaliser des schémas très précis.
\end{minipage}\hfill
\begin{minipage}{0.30\linewidth}
\centering
  \includegraphics[width=3cm]{images/logos/draw_io.png}
\end{minipage}

\vskip 0.5cm

\par Pour conclure ces préliminaires, nous avons mis tout au long de notre projet l'accent sur le travail coopératif via l'utilisation de plateformes proposant de tels services. 

%%% Local Variables: 
%%% mode: latex
%%% TeX-master: "CompteRendu"
%%% End: 
