\par \emph{OAR} est une interface permettant de demander l'accès à des noeuds de calcul (ou en anglais un \emph{resources manager}. L'intérêt d'une telle couche est de ne pas avoir à vérifier quels noeuds sont inutilisés, une commande spécifiant le type de tache à effectuer suffit : interactif ou par batch.


\subsubsection{Utilisation basique}
\label{subsec:utilisation-basique}

\par \emph{OAR} peut être utilisé simplement à l'aide de la commande :
\begin{verbatim}
> oarsub -I
\end{verbatim}

\par On aura demandé un noeud pour 2 heures (le temps par défaut), en interactif, c'est-à-dire qu'une fois demandé et acquis, on utilise ce noeud comme on veut à l'aide du \emph{shell} sur lequel on arrive.

\par Par exemple, 
\begin{verbatim}
-bash-4.2$ oarsub -I
[ADMISSION RULE] Set default walltime to 7200.
[ADMISSION RULE] Modify resource description with type constraints
OAR_JOB_ID=1131
Interactive mode : waiting...
Starting...

Connect to OAR job 1131 via the node sh16
-bash-4.2$ hostname
sh16.grid.metz.supelec.fr
\end{verbatim}
\par On a ainsi accès au noeud demandé, via sh16 pendant 2 heures. Le job représente notre commande, et est identifié par le numéro 1131. Ce numéro va permettre de le manipuler, d'avoir des informations dessus, et éventuellement de le supprimer.
\par On note bien qu'une fois le job acquis, on est connecté sur le noeud sh16 (ce qu'on sait grâce à la commande \texttt{hostname})

\par On peut avoir plus d'informations sur les jobs actifs :
\begin{verbatim}
> oarstat
\end{verbatim}

\par Dans notre exemple, cela renvoie quelque chose comme ça :
\begin{verbatim}
-bash-4.2$ oarstat
Job id     Name           User           Submission Date     S Queue
---------- -------------- -------------- ------------------- - ----------
1126                      munozperez_jua 2014-02-21 15:57:17 R default   
1132                      careil_ale     2014-02-21 16:29:29 R default   
\end{verbatim}

\par D'autres options bien utiles sont disponibles et permettent d'obtenir des informations détaillées à propos des jobs, en voici une liste non exhaustive : \\

\begin{table}[h!]
  \centering

  \begin{tabular}{ll}
    \texttt{-j id\_job} & Permet de spécifier le job qu'on veut étudier. \\
    \texttt{-f} & Affiche la totalité des informations disponibles sur le job (ou tous, si aucun job n'est spécifié).\\
    \texttt{-s} & Affiche l'état d'un job (doit être utilisé avec \texttt{-j}).\\
  \end{tabular}

  \caption{Résumé des options utiles de oarstat}
  
\end{table}


\subsubsection{Plus d'informations sur les jobs actifs}
\label{sec:plus-dinf-sur}

\par Même si \texttt{oarstat} peut permettre d'accéder à beaucoup d'informations, certaines sont déjà accessibles via des variables d'environnement. Cela permet d'éviter de recevoir un tas d'informations dans lequel se trouve l'information précisé que l'on recherchait.

\par Une variable d'environnement qui porte le nom \texttt{VARIABLE} dévoilera son contenu avec un dollar \texttt{\$} devant : \texttt{\$VARIABLE}.

\par Voici une liste des principales variables d'environnement utilisées par \texttt{OAR} lorsqu'un job est actif :
\begin{itemize}
\item OAR\_JOBID : numéro du job.
\item OAR\_WORKING\_DIRECTORY : répertoire initial de soumission.
\item OAR\_NODEFILE : fichier contenant les noeuds utilisés. 
\end{itemize}


%%% Local Variables: 
%%% mode: latex
%%% TeX-master: "CompteRendu"
%%% End: 
