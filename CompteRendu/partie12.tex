\section{Description du projet}

\par Le monde de l’IT fait actuellement parlé de lui à travers le Big Data. Google Trends rend compte de l’explosion l’emploi de ce terme depuis fin 2010. Cette terminologie qui signifie littéralement en Français, «Grandes Données» regroupe l’ensemble des technologies mises en œuvre afin de traiter de gros volumes de données dans le but d’optimiser les process, améliorer l’expérience utilisateur, proposer de nouveaux services et bien plus encore. Les applications du Big Data sont nombreuses et s’appuient pour une grande majorité sur les données opensource mises en ligne par les gouvernements, les entreprises et les particuliers.

\par La deuxième année d’études à Supélec est ponctuée par la réalisation d’un projet logiciel et d’un projet de conception se déroulant chacun au cours d’une séquence (deux mois).

\par Le choix des sujets est libre. C’est pourquoi nous avons choisi de nous intéresser aux technologies Big Data et plus particulièrement à l’environnement Hadoop. Cet outil, très en vogue actuellement et permettant de traiter des gros volumes de données afin d’en tirer des résultats pertinents apparaît comme la référence dans le domaine. 

\par Notre projet s’articule en deux grandes parties, chacune réalisée au cours d’une séquence. Premièrement, il s’agit d’une part de coder une interface graphique utilisateur (en anglais GUI pour Graphique User Interface) à travers laquelle un utilisateur est en mesure de se connecter aux serveurs de Supélec ; d’autre part, il s’agit de permettre à l’utilisateur d’allouer une certaine puissance de calcul au sein des clusters Supélec (Skynet, Cameron et InterCel). Puissance qui, dans une deuxième partie, sera utilisée afin d’exécuter des scripts Hadoop qui auront été codés au préalable. Cette dernière partie se concentrera exclusivement sur Hadoop et notamment sera l’occasion de travailler plus en étroit avec le Data Mining.

%%% Local Variables: 
%%% mode: latex
%%% TeX-master: "CompteRendu"
%%% End: 
